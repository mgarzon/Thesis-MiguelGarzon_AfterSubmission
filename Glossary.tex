\chapter{Glossary}
\label{chap:glossary}

\begin{description}
\item[Umple] 
A technology for adding UML abstractions to code; a portmanteau of `UML Programming Language', `Simple' and `Ample'.
\item[UML]
Unified Modeling Language. A standardized, general-purpose visual modeling language including a set of graphic notations to create models of object-oriented software systems.
\item[Umplification]
An incremental reverse engineering technique that consists in the conversion of software systems written in a base programming language into Umple.
\item[Refactoring]
Improving the quality of software systems by restructuring in small well-defined increments that preserve behaviour.
\item[API]
Application Programming Interface. Is a specification of how some software components should interact with each other.
\item[Model Transformation]
A program that mutates one model into another.
\item[Model]
Simplified representation of a system that helps to gain a better understanding of the system, and in some cases can be used to generate code.
\item[Metamodel]
A model of a modeling language. A metamodel of a model X describes the structure that model X must follow to be valid.
\end{description}

